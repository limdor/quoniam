\hypertarget{_heuristic_measure_8cpp}{\section{src/viewpoint-\/measures/\+Heuristic\+Measure.cpp File Reference}
\label{_heuristic_measure_8cpp}\index{src/viewpoint-\/measures/\+Heuristic\+Measure.\+cpp@{src/viewpoint-\/measures/\+Heuristic\+Measure.\+cpp}}
}
{\ttfamily \#include \char`\"{}Heuristic\+Measure.\+h\char`\"{}}\\*
{\ttfamily \#include \char`\"{}Tools.\+h\char`\"{}}\\*
